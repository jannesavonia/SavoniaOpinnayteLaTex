\input{savonia.inc}
\newcommand{\theauthor}{
  Tekijän Nimi -- Täytä \texttt{perustiedot.tex}
}

\newcommand{\thetitle}{
  Opinnäytetyön pohja
}

\newcommand{\thesubtitle}{
  Alaotsikko
}

\newcommand{\reporttype}{
  Raportin tyyppi (Opinnäytetyö?)
}

\newcommand{\schoolfi}{
  Koulutusala suomeksi
}

\newcommand{\schooluk}{
  Koulutusala englanniksi
}

\newcommand{\degreefi}{
  Tutkinto suomeksi
}

\newcommand{\degreeuk}{
  Tutkinto englanniksi
}

\newcommand{\publishdate}{
  \today
} %Julkaisupäivä, voidaan asettaa myös käsin

\newcommand{\totalappendix}{
  Täytä liitteiden lukumäärä tänne
}


%%%%%%%%%%%%%%%%%%%%%%%%%%%%%%%%%%%%%%%%%%%%%%%%%%%%%%%%%%%%%%%%
%Älä muokkaa alla olevia
\author{\theauthor}
\title{\thetitle}
\newcommand{\totalpages}{\pageref{LastPage}}


%%%%%%%%%%%%%%%%%%%%%%%%%%%%%%%%%%%%%%%%%%%%%%%%%%%%%%%%%%%%%%%%%%%
%usepackages etc. here

\usepackage{hyperref}


%%%%%%%%%%%%%%%%%%%%%%%%%%%%%%%%%%%%%%%%%%%%%%%%%%%%%%%%%%%%%%%%%%%
\begin{document}
%Create cover page and table of contents
\input{firstpages.inc}

\section{Johdanto}
\begin{itemize}
\item Päätiedosto on \texttt{opinnayte.tex}. Tiedoston nimen voi vaihtaa.
\item \texttt{perustiedot.tex} sisältää työn perustiedot, kuten opiskelijan nimi, työn nimi jne. Muokkaa tiedosto kuntoon.
\item Kirjoita tiivistelmä tiedostoon \texttt{tiivistelma.tex}.
\item Kirjoita englanninkielinen tiivistelmä tiedostoon \texttt{abstract.tex}.
\item Täytä työn avainsanat \texttt{avainsanat.tex}-tiedostoon.
\item Täytä työn avainsanat englanniksi \texttt{keywords.tex}-tiedostoon.
\item Kirjaa työn yhteistyökumppanit ja tilaaja tiedstoon \texttt{yhteistyokumppanit.tex}.
\item Pohja on saatavilla githubissa \href{https://github.com/jannesavonia/SavoniaOpinnayteLaTex}{\texttt{https://github.com/jannesavonia/SavoniaOpinnayteLaTex}}. 
\end{itemize}

\subsection{Aliluku}

\subsubsection{Alialiluku}

\section{Toinen luku}

\pagebreak
\appendix

\section{ENSIMMÄINEN LIITE}

\section{TOINEN LIITE}

\end{document}
